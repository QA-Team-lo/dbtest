%!TeX program = xelatex
\documentclass{article}
\usepackage{papers}

%%-------------------------------正文开始---------------------------%%
\begin{document}

%%-----------------------封面--------------------%%
\maketitle

%%------------------摘要-------------%%
\begin{abstract}
本次测试报告旨在验证 TiDB 数据库在 RISC-V 平台上的可用性,并进行了 TiDB 在 Milk-V Pioneer Box 和 Sipeed LicheePi 4A 两个 RISC-V 典型平台上的测试。
测试结果待补。
\end{abstract}

\thispagestyle{empty} % 首页不显示页码

%%--------------------------目录页------------------------%%
\newpage
\tableofcontents

%%------------------------正文页从这里开始-------------------%
\newpage

\begin{markdown}

# 简介

## 软件说明
TiDB 是一个开源的分布式 SQL 数据库,由 PingCAP 开发和维护。它兼容 MySQL 协议,具有水平扩展、高可用和强一致性的特点,广泛应用于金融、互联网、制造等行业。

## 测试目的
本次测试旨在验证 TiDB 在 RISC-V 平台上的可用性,特别是在 Milk-V Pioneer Box 和 Sipeed LicheePi 4A 两个典型平台上的表现。本报告通过手动测试的方法,从目前的平台兼容性及性能测试两个角度评估了 TiDB 当前在 RISC-V 平台上的可用性,并给出了定性和定量的结论,为其未来进一步的优化和支持提供参考。

## 测试概述
本次测试在 RISC-V 设备 Milk-V Pioneer Box 和 Sipeed LicheePi 4A 的多个 Linux 发行版上对多个版本的 TiDB 进行了测试。对其目前在 RISC-V 上的可用性进行了较为全面的测试并得出了相应的结论。

本报告在部分 Linux 发行版下还使用了基准测试以测试数据库的性能。

## 测试总结
本次测试在以下平台和系统上验证情况如下:

| 平台        | 发行版        | 测试结果              |
|-------------|---------------|-----------------------|
| Pioneer Box | RevyOS        | TBD                   |
| LicheePi 4A | RevyOS        | TBD                   |
| Pioneer Box | openEuler     | TBD                   |
| LicheePi 4A | openEuler     | TBD                   |
| LicheePi 4A | Fedora        | TBD                   |
| Pioneer Box | Fedora        | TBD                   |
| Both        | openKylin     | TBD                   |
| Both        | openCloudOS   | TBD                   |
| Both        | Ubuntu        | TBD                   |
| Both        | Debian        | TBD                   |

性能测试结果如下:

- 基准测试:

| Pioneer Box | LicheePi 4A | x86-64 参考 |
|-------------|-------------|-------------|
|    TBD      | TBD         | TBD         |

# 环境说明

## 硬件环境
本次测试主要在 Milk-V Pioneer Box 和 Sipeed LicheePi 4A 上进行,机器硬件配置为:

Milk-V Pioneer Box:
- CPU: SG2042 64 Core C920@2.0GHz
- RAM: 4 channel 3200Hz 128GB DDR4 SODIMM (32GB * 4)
- SSD: PCIe 3.0 x 4 1TB
- GPU: AMD R5 230

Sipeed LicheePi 4A:
- CPU: TH1520, RISC-V 2.0G C910 x4
- RAM: 16 GB 64bit LPDDR4X-3733
- Storage: 128 GB eMMC

## 软件环境

本次测试涵盖的系统版本和 TiDB 版本如下:
- openEuler: 第三方软件仓库自带 TiDB 
  - Pioneer Box 版本: TiDB TBD
  - LicheePi 4A 版本: (未测试)
- RevyOS: 软件仓库自带 TiDB
  - Pioneer Box 版本: TiDB TBD
  - LicheePi 4A 版本: TiDB TBD
  其余未测试系统及原因如下:

  | 平台        | 发行版        | 未测试原因              |
  |-------------|---------------|-----------------------|
  | LicheePi 4A | openEuler     | TBD                   |
  | LicheePi 4A | Fedora        | TBD                   |
  | Pioneer Box | Fedora        | TBD                   |
  | Both        | openKylin     | TBD                   |
  | Both        | openCloudOS   | TBD                   |
  | Both        | Ubuntu        | TBD                   |
  | Both        | Debian        | TBD                   |

## 环境搭建

...

# 总结

本次报告评估了 TiDB 数据库在 RISC-V 平台上的可用性和性能。

\end{markdown}

\newpage
\section{附录}

\appendix

以下是本报告使用的手动测试用例,结果见 \href{https://github.com/QA-Team-lo/dbtest}{Github 仓库}。

\begin{itemize}
    \item ...
\end{itemize}

\reference

\end{document}
