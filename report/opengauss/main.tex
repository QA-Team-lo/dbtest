%!TeX program = xelatex
\documentclass{article}
\usepackage{papers}

%%-------------------------------正文开始---------------------------%%
\begin{document}

%%-----------------------封面--------------------%%
\maketitle

%%------------------摘要-------------%%
\begin{abstract}
本次测试报告旨在验证 openGauss 数据库在 RISC-V 平台上的可用性,并进行了 openGauss 在 Milk-V Pioneer Box 和 Sipeed LicheePi 4A 两个 RISC-V 典型平台上的测试。测试涵盖了多个 Linux 发行版和 openGauss 版本,通过手动测试和性能测试评估其在 RISC-V 平台上的可用性。手动测试通过模拟用户日常使用场景对可用性进行验证,测试显示 openGauss 在 RevyOS Linux 发行版上具备基本的数据库操作能力,而在 openEuler 上存在稳定性问题。性能评估结果显示,RISC-V 平台上的 openGauss 在 sysbench 基准测试中表现不佳,尤其在需要高性能的数据库操作任务上。受限于硬件和软件情况,其他系统无法完整运行或缺少支持。这表明 openGauss 已初步支持 RISC-V 平台,但在功能完整性和性能优化上仍需改进。
\end{abstract}

\thispagestyle{empty} % 首页不显示页码

%%--------------------------目录页------------------------%%
\newpage
\tableofcontents

%%------------------------正文页从这里开始-------------------%
\newpage

\begin{markdown}

# 简介

## 软件说明
openGauss 是一个免费的开源关系型数据库管理系统,主要由华为开发和维护。它是一个广泛使用的代码库,为企业级应用提供了高性能、高可用性和高安全性的数据库解决方案。

## 测试目的
本次测试旨在验证 openGauss 在 RISC-V 平台上的可用性,特别是在 Milk-V Pioneer Box 和 Sipeed LicheePi 4A 两个典型平台上的表现。本报告通过手动测试的方法,从目前的平台兼容性及用户的日常使用体验两个角度评估了 openGauss 当前在 RISC-V 平台上的可用性,并给出了定性和定量的结论,为其未来进一步的优化和支持提供参考。

## 测试概述
本次测试在 RISC-V 设备 Milk-V Pioneer Box 和 Sipeed LicheePi 4A 的多个 Linux 发行版上对多个版本的 openGauss 进行了 sysbench 测试。对其目前在 RISC-V 上的可用性进行了较为全面的测试并得出了相应的结论。

本报告在部分 Linux 发行版下还使用了 sysbench 以测试数据库的性能。

## 测试总结
本次测试在以下平台和系统上验证情况如下:

| 平台        | 发行版        | 测试结果              |
|-------------|---------------|-----------------------|
| Pioneer Box | RevyOS        |                       |
| LicheePi 4A | RevyOS        |                       |
| Pioneer Box | openEuler     |                       |
| LicheePi 4A | openEuler     |                       |
| LicheePi 4A | Fedora        |                       |
| Pioneer Box | Fedora        |                       |
| Both        | openKylin     |                       |
| Both        | openCloudOS   |                       |
| Both        | Ubuntu        |                       |
| Both        | Debian        |                       |

性能测试结果如下:

- sysbench:

| Pioneer Box | LicheePi 4A | x86-64 参考 |
|-------------|-------------|-------------|
|    TBD      |    TBD      |     TBD     |

其余未测试系统及原因如下:

# 环境说明

## 硬件环境
本次测试主要在 Milk-V Pioneer Box 和 Sipeed LicheePi 4A 上进行,机器硬件配置为:

Milk-V Pioneer Box:
- CPU: SG2042 64 Core C920@2.0GHz
- RAM: 4 channel 3200Hz 128GB DDR4 SODIMM (32GB * 4)
- SSD: PCIe 3.0 x 4 1TB
- GPU: AMD R5 230

Sipeed LicheePi 4A:
- CPU: TH1520, RISC-V 2.0G C910 x4
- RAM: 16 GB 64bit LPDDR4X-3733
- Storage: 128 GB eMMC

## 软件环境

本次测试涵盖的系统版本和 openGauss 版本如下:

https://gitee.com/opengauss/riscv 6.0.0

## 环境搭建

### 安装系统

#### Sipeed LicheePi 4A

LicheePi 4A 各个系统在[支持矩阵](https://github.com/ruyisdk/support-matrix/tree/main/LicheePi4A)上详细记载了安装过程,可作为参考。

- RevyOS
从 [ISCAS 镜像](https://mirror.iscas.ac.cn/revyos/extra/images/lpi4a/)下载并解压:
```shell!
wget https://mirror.iscas.ac.cn/revyos/extra/images/lpi4a/20240720/boot-lpi4a-20240720_171951.ext4.zst
wget https://mirror.iscas.ac.cn/revyos/extra/images/lpi4a/20240720/u-boot-with-spl-lpi4a.bin
wget https://mirror.iscas.ac.cn/revyos/extra/images/lpi4a/20240720/root-lpi4a-20240720_171951.ext4.zst
zstd -d boot-lpi4a-20240720_171951.ext4.zst
zstd -d root-lpi4a-20240720_171951.ext4.zst
```

按住 boot 键后,上电/Reset 进入刷写模式。

刷写系统:
```shell!
sudo fastboot devices
sudo fastboot flash ram u-boot-with-spl-lpi4a.bin 
sudo fastboot reboot
sudo fastboot flash uboot u-boot-with-spl-lpi4a.bin
sudo fastboot flash boot boot-lpi4a-20240720_171951.ext4
sudo fastboot flash root root-lpi4a-20240720_171951.ext4
```
默认账号密码为:`debian`:`debian`

#### Milk-V Pioneer Box

##### RevyOS
下载 [RevyOS 20241025](https://mirror.iscas.ac.cn/revyos/extra/images/sg2042/20241025/revyos-pioneer-20241025-001347.img.zst)后,使用 `zstd` 解压, `dd` 到 **NVMe 硬盘**中 *(需要一个硬盘盒)*
准备 microSD 读卡器和存储卡,下载 [固件](https://mirror.iscas.ac.cn/revyos/extra/images/sg2042/20241025/firmware_single_sg2042-v6.6-lts-v0p7.img) 并 `dd` 到存储卡中。
```shell!
zstd -d revyos-pioneer-20241025-001347.img.zst
sudo dd if=path/to/revyos-pioneer-20241025-001347.img of=/dev/your/nvme bs=4M status=progress
sudo dd if=path/to/firmware_single_sg2042-v6.6-lts-v0p7.img of=/dev/your/sdcard bs=4M status=progress
```
将存储卡和硬盘插入系统后上电开机。

##### openEuler
下载[系统镜像](https://mirrors.hust.edu.cn/openeuler/openEuler-24.03-LTS/embedded_img/riscv64/SG2042/openEuler-24.03-LTS-riscv64-sg2042.img.zip),解压,使用 `dd` 烧录至 NVMe 硬盘。
下载[固件](https://mirrors.hust.edu.cn/openeuler/openEuler-24.03-LTS/embedded_img/riscv64/SG2042/sg2042_firmware_linuxboot.img.zip),解压,使用 `dd` 烧录至 microSD 卡。

请将下面的 `/dev/sda` `/dev/sdb` 替换成实际使用的硬盘和存储卡位置。

```shell!
unzip openEuler-24.03-LTS-riscv64-sg2042.img.zip
sudo wipefs -af /dev/sda
sudo dd if=openEuler-24.03-LTS-riscv64-sg2042.img of=/dev/sda bs=1M status=progress
sudo eject /dev/sda
unzip sg2042_firmware_linuxboot.img.zip
sudo dd if=sg2042_firmware_linuxboot.img of=/dev/sdb bs=1M status=progress
```
将存储卡和硬盘插入系统上电开机。

### 手动测试环境

#### RevyOS

...

#### openEuler

...

### 性能测试

...

# 测试内容

## 手动测试

...

## 性能测试

...

# 测试结果

详细测试数据可见 [Github 仓库](https://github.com/QA-Team-lo/chromium_test)

## 手动测试

...


## 性能测试

...

# 总结

本次报告评估了 openGauss 数据库在 RISC-V 平台,特别是 Milk-V Pioneer Box 和 Sipeed LicheePi 4A 上的可用性和性能。

...

这些结果表明,尽管 openGauss 在 RISC-V 平台上已具备一定可用性,但仍需进一步优化和支持,以提高其功能完整性和性能表现。

\end{markdown}

\newpage
\section{附录}

\appendix

以下是本报告使用的测试用例,结果见 \href{https://github.com/QA-Team-lo/dbtest}{Github 仓库}。

\begin{itemize}
    \item (待补)
\end{itemize}

\reference

\end{document}