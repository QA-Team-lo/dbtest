%!TeX program = xelatex
\documentclass{article}
\usepackage{papers}

%%-------------------------------正文开始---------------------------%%
\begin{document}

%%-----------------------封面--------------------%%
\maketitle

%%------------------摘要-------------%%
\begin{abstract}
本次报告旨在验证 OceanBase 数据库在 RISC-V 平台上的可用性。通过官方文档和社区尝试,发现 OceanBase 当前不支持 RISC-V 架构。尝试编译 OceanBase 代码时,系统会检测到 RISC-V 架构并报错。社区开发者提交了相关 Pull Request,但仍未能成功编译。这表明 OceanBase 在 RISC-V 平台上的支持仍需进一步开发和优化。
\end{abstract}

\thispagestyle{empty} % 首页不显示页码

%%--------------------------目录页------------------------%%
\newpage
\tableofcontents

%%------------------------正文页从这里开始-------------------%
\newpage

\begin{markdown}

# 简介

## 软件说明
OceanBase 是一个分布式关系数据库系统,主要由蚂蚁集团开发和维护。它旨在提供高可用性、高性能和高扩展性的数据库解决方案,广泛应用于金融、电商等领域。

## 测试目的
本次测试旨在验证 OceanBase 数据库在 RISC-V 平台上的可用性。通过官方文档和社区尝试,评估其在 RISC-V 平台上的编译和运行情况,并为未来的优化和支持提供参考。

## 测试概述
本次测试通过官方提供的编译工具链和社区开发者的尝试,评估了 OceanBase 在 RISC-V 平台上的编译情况。测试发现,官方文档明确指出当前不支持 RISC-V 架构。尝试编译 OceanBase 代码时,系统会检测到 RISC-V 架构并报错。社区开发者提交了相关 PR,但仍未能成功编译。

## 测试总结

本次测试结果如下:

| 平台        | 发行版        | 测试结果              |
|-------------|---------------|-----------------------|
| RISC-V      |  Ubuntu 22.04      | 无法构建,不支持        |
| RISC-V      |  Any         | 预计同样无法构建,不支持 |

# 环境说明

## 硬件环境
本次测试主要在 RISC-V 平台上进行,机器硬件配置为:

Milk-V Pioneer Box:
- CPU: SG2042 64 Core C920@2.0GHz
- RAM: 4 channel 3200Hz 128GB DDR4 SODIMM (32GB * 4)
- SSD: PCIe 3.0 x 4 1TB
- GPU: AMD R5 230

## 软件环境

openEuler 24.03 LTS, 使用 Docker 容器运行 Ubuntu 22.04 

## 环境搭建

### 安装系统

#### Milk-V Pioneer Box

下载[系统镜像](https://mirrors.hust.edu.cn/openeuler/openEuler-24.03-LTS/embedded_img/riscv64/SG2042/openEuler-24.03-LTS-riscv64-sg2042.img.zip),解压,使用 `dd` 烧录至 NVMe 硬盘。

下载[固件](https://mirrors.hust.edu.cn/openeuler/openEuler-24.03-LTS/embedded_img/riscv64/SG2042/sg2042_firmware_linuxboot.img.zip),解压,使用 `dd` 烧录至 microSD 卡。

实际操作时,请将下面的 `/dev/sda`、`/dev/sdb` 替换成真实的硬盘和存储卡位置。

```shell
unzip openEuler-24.03-LTS-riscv64-sg2042.img.zip
sudo wipefs -af /dev/sda
sudo dd if=openEuler-24.03-LTS-riscv64-sg2042.img of=/dev/sda bs=1M status=progress
sudo eject /dev/sda
unzip sg2042_firmware_linuxboot.img.zip
sudo dd if=sg2042_firmware_linuxboot.img of=/dev/sdb bs=1M status=progress
```
将存储卡和硬盘插入系统上电开机。

### 编译 OceanBase

#### 官方工具链

尝试安装官方提供的依赖列表,所有依赖均可正常安装。

```bash
sudo apt install git wget rpm rpm2cpio cpio make build-essential binutils m4
```

尝试使用主仓库中的 `build.sh` 进行编译,会检测到 RISC-V 架构并不在支持列表中。

编译随即中止,并产生如下错误信息。

```text
[dep_create.sh] [ERROR] 'Ubuntu 22.04.5 LTS (riscv64)' is not supported yet.
```

#### 社区尝试

社区开发者提交了  [Pull Request WIP: make OB a generic cmake project #847](https://github.com/oceanbase/oceanbase/pull/847) ,试图使 cmake 在 OceanBase 构建上可用。

但采用 [loongarch64/oceanbase](https://github.com/loongarch64/oceanbase) 分支的最新 commit 编译,同样无法得到 RISC-V 架构的可执行产物。


# 总结

本次报告评估了 OceanBase 数据库在 RISC-V 平台上的可用性和性能。通过测试发现,OceanBase 当前不支持 RISC-V 架构。尝试编译 OceanBase 代码时,系统会检测到 RISC-V 架构并报错。尽管社区开发者提交了相关 Pull Request,但仍未能成功编译。这表明 OceanBase 在 RISC-V 平台上的支持仍需进一步开发和优化。

\end{markdown}

\newpage
\section{附录}

\appendix

此处是本报告的附录。

其他数据库测试内容可见本报告的 \href{https://github.com/QA-Team-lo/dbtest}{Github 仓库}。

% TODO: 也许需要添加的更多附录信息

\reference

\end{document}
